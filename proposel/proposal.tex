\documentclass[]{hci-proposal}

% Select your language
%% \usepackage[ngerman]{babel}
\usepackage[english]{babel}

\title{Interaktiver Reiseführer}
\author{Anastasia Parlapani, Sandra Menhorn, Pascal Leber, Daria Wahl, Christoph Meyer}
\date{\today}
\email{your.email@uni-ulm.de}

\begin{document}

% Keine Seitenzahlen
\pagenumbering{gobble}

\maketitle

\begin{abstract}
  Wir entwickeln einen interaktiven Reiseführer für die Stadt Ulm. Mithilfe eines Smartphones soll
  der Nutzer die Stadt Ulm erkunden. Auf einer Stadtkarte sind verschiedene Stationen vermerkt, die
  der Nutzer ansteuern kann. Dort trifft er auf berühmte Ulmer Persönlichkeiten, mit denen der er
  interagieren kann und verschiedene Aufgaben in Form vom Mini-Games absolvieren muss. 
  
  
  
\end{abstract}


\section{Zielgruppe}
Unsere Zielgruppe sind Schüler und Jugendlich ab 12 Jahren, weswegen wir auf einen Edutainment Ansatz befolgen.


\section{Problem, Goal and Approach}
Klassische Reiseführer in Buchform bieten für eine jüngere Zielgruppe keinen Mehrwert.
Als Rahmenhandlung erzählen wir von dem Missglückten Physikversuch des fiktiven Professors Alfred Zweistein weswegen vergangene Persönlichkeiten der Ulmer Stadt Geschichte in unserer Zeit geraten sind.
Er schickt den Nutzer als seinen Assistenten in die Stadt um Albert Einstein zu finden welcher ihm bei dem Problem helfen soll.
Für diese Aufgabe muss der User andere Berühmtheiten befragen um von diesen den Standort zu erfahren.


\section{Previous Work}


\section{Implementation}
Implementiert wird der Reiseführer als Android Application. Um den Nutzer eine bessere Emersion  zu vermitteln werden Standortdaten Abgefragt und bei erreichen von vordefinierten Lokalitäten wird ein Animationsfilm abgespielt.
Als Programmiersprache wird 
\section{Evaluation}
How do you test your work.
What needs to be tested to prove your contribution as relevant.

\section{Time Schedule}
See this list as examples, feel free to add more if needed.
\begin{itemize}
\item Literature Research
\item Preparation (e.g. Hardware Setup)
\item Framework/Architecture Development
\item Implementation
\item Evaluation
\item Study
\item Writing
\item (Gantt-Chart to illustrate the time schedule)
\end{itemize}

\bibliographystyle{IEEEtranS}   % alternativer Stil
\bibliography{bibliography} % Bib-Datei

\begin{landscape}
  \begin{figure*}[htbp]
    \begin{ganttchart}[vgrid, hgrid, y unit chart=0.7cm]{1}{24}
      % Titles
      \gantttitle{2014}{12}
      \gantttitle{2015}{12} \\
      \gantttitlelist{1,...,24}{1} \\

      % First Group
      \ganttgroup{Group 1}{1}{10} \\
      \ganttbar{Literature Research}{1}{3} \\
      \ganttbar{Preparation }{4}{10} \\
      \ganttbar{Implementation}{9}{12} \\
      \ganttmilestone{Milestone 1}{8} \\
      \ganttlink{elem1}{elem2}
      \ganttlink{elem2}{elem3}

      % Second Group
      \ganttgroup{Group 2}{13}{21} \\
      \ganttbar{Task 1}{13}{18} \\
      \ganttbar{Task 2}{18}{21} \\
      \ganttbar{Task 3}{22}{24} \\
      \ganttmilestone{Milestone 2}{17}
      \ganttlink{elem6}{elem7}
      \ganttlink{elem7}{elem8}
    \end{ganttchart}
    \label{gc}
    \caption{Gantt-Chart example}
  \end{figure*}
\end{landscape}

\end{document}
